  
\documentclass{article}
\usepackage[T1]{fontenc}
\usepackage[utf8]{inputenc}
\usepackage[document]{ragged2e}
\usepackage{tgbonum}
\usepackage{graphicx} 
\usepackage{hyperref}
\hypersetup{
    colorlinks=true,
    linkcolor=blue,
    filecolor=magenta,      
    urlcolor=cyan,
}
\urlstyle{same}
\usepackage{geometry}
 \geometry{
 a4paper,
 total={170mm,257mm},
 left=20mm,
 top=20mm,
 }
%\documentclass{beamer}
%\usetheme{Madrid}
%\usecolortheme{beaver}



 
\graphicspath{ {C:\Users\Administrator.Ganesh-PC\Desktop\project}{C:\Users\Administrator.Ganesh-PC\Desktop\project} }
 
\newpage
\begin{document}
\begin{center}
    


\thispagestyle{empty}

\LARGE{\textsc {\textbf{PLASTIC BOTTLE RECYCLING}}}\\[0.1cm]
\LARGE{\textsc {\textbf{FOR COMMERCIAL AND}}}\\[0.1cm]
\LARGE{\textsc {\textbf{DOMESTIC PURPOSE}}}\\[0.5cm]


\vspace{0.5cm}

\Large{\textbf{\\BY}}\\[0.3cm]
\begin{table}[h]
\centering
\Large{\bfseries TEAM-ID:45}\\[0.3cm]
\Large{
\begin{tabular}{l c r}
{\bfseries GROUP MEMBER 1}& & {\bfseries Soham Sawant}\\{\bfseries GROUP MEMBER 2} & & {\bfseries Ronak Upasham}\\{\bfseries GROUP MEMBER 3} & &{\bfseries Ganeshprasad Keni}\\{\bfseries GROUP MEMBER 4} & &{\bfseries Heron Vas}\\
\end{tabular}}
\end{table}
\vspace{0.5cm}
\Large{\textbf{UNDER THE GUIDANCE OF}}\\
\Large{\textbf{Prof. Kranti Wagle}}\\
\vspace{1cm}
\large{\textbf{}}\\
\Large{\textbf{Fr. Conceicao Rodrigues College Of Engineering}}\\
\Large{\textbf{Bandra (W), Mumbai -400050 }}
\Large{\textbf{\\2019-2020}}\\
\vspace{1cm}
\Large{\textbf{\\}}
\LARGE{\textbf{}}
\newpage
\end{center}

\newpage

\begin{center}
    

    
    
    \LARGE{\textsc {\textbf{SUSTAINABLE DEVELOPEMENT GOALS}}}\\[0.5cm]
    
\end{center}

\subsection{LIFE ON EARTH} % The \section*{}} command stops section numbering
\Large
It leads to land pollution which affects human life and animal life toxic gases leads out of plastic are found in human tissues and blood which remains in it leading to the problems like cancer it is also poisonous to animals animals like cows are affected by it by the consumption of plastic unknowingly.\newline
\subsection{UNDERWATER LIFE} % The \section*{}} command stops section numbering
\Large
It also affects the marine life causing the blockage problem and can be poisonous for the marine life too.\newline

\newpage

\begin{center}
    \LARGE{\textsc {\textbf{CUSTOMER AND MARKET RESEARCH }}}\\[0.5cm]
\end{center}
\begin{itemize}
\Large
    \item The technology we use should be very efficient for the user as well as for the producer so that it can be implemented easily.
 \item We came to know about various problem that our industries face regarding the proper management of plastic waste.
\item We also came to know about the different solutions regarding the problem.
\item We came across various solutions for the development of the product which were proposed by them 
\end{itemize}
\vspace{0.5cm}
\begin{center}
    \LARGE{\textsc {\textbf{ATOMIC UNIT(KEY FEATURES)}}}\\[0.5cm]
\end{center}
\begin{flushleft}
\Large
    compactness, easy to use,less power consumption and portable
\end{flushleft}
\vspace{0.5cm}
\begin{center}
    \LARGE{\textsc {\textbf{KEY PERFORMANCE INDICATOR}}}\\[0.5cm]
\end{center}
\begin{itemize}
\Large
    \item \textbf{User Interface-}It makes the product user \\interactive and will get more and more users to use it. 
\item \textbf{Shredding and Melting-}It performs some \\activities assigned which then plays an \\important role in the recycling process.
\end{itemize}
\newpage

\begin{center}
    
    \LARGE{\textsc {\textbf{HOW MIGHT WE?}}}\\[0.5cm]
    
\end{center}

\Large
\justifying
We provide a simple solution in the form of a plastic bottle recycling reverse vending machine. It is an integrated approach to the processes of plastic bottle disposal and recycling. A single unit capable of collecting waste plastic bottles, shredding them into small pieces to facilitate the injection molding process to remold the plastic later on into reusable products is implemented in a compact space. We aim to complement the current expensive and tedious system of plastic bottle recycling with a cheaper, easily scalable, and a more accessible piece of machinery to expedite the global effort to curb plastic pollution.

\vspace{2cm}
\begin{center}
    
    \LARGE{\textsc {\textbf{CAUSES}}}\\[0.5cm]
    
\end{center}

\begin{itemize}

    \item Inefficient Disposal System of Plastic Bottles
    \item Less Accessibility of Industrial Methods 
    \item Careless attitude of People
\item Ineffective Government Policies 
\item Overusage of Plastic
\end{itemize}

%\newpage

\begin{center}
    
    \LARGE{\textsc {\textbf{EFFECTS}}}\\[0.5cm]
    
\end{center}

\begin{itemize}
\Large
    \item Inefficient Disposal System of Plastic Bottles
    \item Less Accessibility of Industrial Methods 
    \item Careless attitude of People
\item Ineffective Government Policies 
\item Overusage of Plastic
\end{itemize}
\vspace{1cm}
\begin{center}
    \newpage
    \LARGE{\textsc {\textbf{STAKEHOLDERS}}}\\[0.5cm]
    
\end{center}

\begin{itemize}
\Large
    \item Common People
\item Ecosystem
\item Municipal Corporation(Civil Workers)
\item Plastic Industries
\item Ragpickers
\item Investors
\end{itemize}

\newpage

\begin{flushleft}
    \ \LARGE{\textsc {\textbf{IMPLEMENTATION}}}\\
\end{flushleft}
\vspace{0.5cm}
\includegraphics[width=16cm, height=23cm]{imp1.jpeg}
\newpage
\section{Shredder}
\Large
Shredder is a tool which is used for shredding the plastic bottles into small pieces which will facilitate the melting process.
Making of shredder requires laser cut parts which are made of Mild Steel and it is designed using the software \textbf{Solidworks}\\(link:-\href{https://www.solidworks.com/sw/support/downloads.htm}{https://www.solidworks.com/sw/support/downloads.htm}). By using a single phase-AC induction motor (1.5 HP, 2500 RPM) we are rotating the shaft of Shredder, whose RPM is reduced by using a 100:1 reduction gear box and the torque is increased to certain threshold which is required to shred the plastic properly.
Now the shredded plastic will proceed to further process. 
\newline
\section{Separation Chamber}
\Large
Depending upon the type of bottle inserted by the user, the shredded plastic particles will be divided into two sections. One will consists of HDPE particles and also other types of plastic material whose melting temperature is in the same range and the other section will be only type of PET particles. Weight sensors will be attached to the chambers
to weigh the amount of plastic particles accumulated . Once the desired amount of plastic is present of a certain type then that type of plastic will be dropped in the injection moulding chamber. This is the done to avoid the presence of plastic having different melting temperature and also to avoid defected product produced due to less amount of material inserted at a time.  
 \newline
\section{Injection Moulding}
\Large
A \textbf{High torque DC motor} is used to rotate the shaft that is connected to Archimedes screw. The Archimedes screw will propel the particles forward in the chamber, the chamber is further heated by two \textbf{230V AC Band heaters}. A \textbf{PID Temperature Control} Algorithm is implemented to control the temperature for the optimal melting point of plastic to avoid burning which also uses phase angle control method of TRIAC to control the output AC voltage to the heaters. The molten plastic is moulded to a form of desired size 3-D printer filaments.
\newline
\section{GUI}
\Large
\textbf{A Graphical User Interface(GUI)} is used to monitor all the functions of the machines to give a proper output. It provides a platform for the user to interact with the machine by using certain visual notations or buttons. This GUI is hosted on a screen where user can access it directly with the help of some buttons or touchscreen technology. It will be made using the following \\technologies:
\begin{itemize}
    	\item Arduino Duo/Mega
	 \item C (language used)
	 \item Arduino IDE (installation link:- \href{https://arduino.en.softonic.com/}{https://arduino.en.softonic.com/})
	\item 7 inch TFT LCD display
\end{itemize}
This section will provide the users the information regarding different types of bottles, number of bottles the users wants to insert and also will provide them to select the dimensions of the bottle that they want to put. It will also provide instruction to the user to remove the label and cap and also empty the fluid inside the  bottle before inserting. It will also display the error message in case if fluid is present in the bottle or label is present.      
 \newline
\section{Coin Vending Machine}
\Large
This system will contain coins of a particular amount and depending upon the type and number of bottles inserted by the user, it will provide the user with small token amount in exchange for the bottles. A mechanism consisting of laser transmitter and suitable receiver and dc motor to dispatch the amount.
\newline
\begin{center}
\includegraphics[width=16cm, height=25cm]{imp2.jpeg}    
\end{center}
\newpage
\section{Feasibility}
\Large
Our proposed project idea is highly suitable to be used in colleges due to the recycled final product which will be of use to the college in which it is placed and provides an avenue for the college to save resources. The machine requires low maintenance. To further expand the idea in future, external molds can be added to the outlet of the extruder to produce different recycled articles depending upon the needs of the stakeholder. It has the potential to be used at public places l like railway stations, tourist spots, etc. which will accelerate the efforts of government to  curb plastic pollution while also reducing requirements of manpower and resources. It lacks a mechanism to clean the bottles which are inserted, hence there is a risk of the machine wearing down due to the presence of foreign particles. The machine is not capable of removing the labels from the bottles, hence care should be taken while inserting the bottles to reduce the possibility of harm to the machine. Care should be taken to insert only the specific type of bottle which will be selected by the user in the GUI, or it may result in plastic being burnt and clogging the whole machine. A continuous supply of electricity is required, hence in case of a power outage the machine will stop working.
\section{References}
\begin{enumerate}
   \item Dave Hakkens- \href{https://davehakkens.nl}{https://davehakkens.nl}   
\item Parley- \href{https://www.parley.tv/#fortheoceans}{https://www.parley.tv/#fortheoceans}
\item Wildwest India –\href{www.wildwestmedia.in}{www.wildwestmedia.in}
\item Rudra Environmental Solutions pvt Ltd-\href{https://rudraenvsolution.com/}{https://rudraenvsolution.com/}
\item Making of 3D printer filament-\href{https://www.instructables.com/id/Build-your-own-3d-printing-filament-factory-Filame/}{https://www.instructables.com/id/Build-your-own-3d-printing-filament-factory-Filame/}
\end{enumerate}
\end{document}


